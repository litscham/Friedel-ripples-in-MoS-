\documentclass[12pt]{article}
\usepackage{geometry}
\usepackage{amsmath, amssymb}
\usepackage{graphicx}
\usepackage{booktabs}
\usepackage{hyperref}
\usepackage{natbib}
\geometry{a4paper, margin=1in}

\begin{document}

\section{Literature Review: Friedel Oscillations in Bulk Materials}

\subsection{Introduction}
Friedel oscillations represent one of the most fundamental phenomena in condensed matter physics, arising from the interaction between conduction electrons and localized perturbations in metallic systems. Named after Jacques Friedel, who first formalized the theory in the early 1950s, these oscillations manifest as spatially periodic variations in electron density surrounding impurities or defects in metals. Friedel oscillations reveal essential insights into electron scattering and screening and influence the electronic, magnetic, and structural properties of bulk materials, alloys, and interfaces \citep{friedel1952electrons}.

The objective of this review is to provide a comprehensive survey of classical Friedel oscillation theory and experimental evidence. Key aspects include an examination of Friedel's original 1952 paper, phase shift analysis, scattering theory, derivations of the Friedel sum rule, and documented experimental observations in various bulk metals and alloys. A comparative analysis of oscillation characteristics across material types is also presented.

\subsection{Theoretical Foundations}

\subsubsection{Friedel's Original Work (1952)}
Friedel's seminal paper, \textit{"The Distribution of Electrons Round Impurities in Monovalent Metals"} (1952), laid the groundwork for understanding electron density oscillations in metals caused by a localized scattering potential. He showed that conduction electrons respond to the perturbation not merely by screening the impurity but by creating a spatially oscillatory modulation of electron density:

\begin{equation}
\Delta \rho(\mathbf{r}) = \rho(\mathbf{r}) - \rho_0 \propto \frac{\cos(2k_F r + \delta)}{r^3} \citep{friedel1952electrons}
\end{equation}

where $k_F$ is the Fermi wavevector, $r$ is the distance from the impurity, and $\delta$ is a phase shift arising from electron scattering.

\subsubsection{Scattering Theory and Phase Shifts}
In scattering theory, an impurity is treated as a localized potential $V(\mathbf{r})$ interacting with free electrons. The electron wavefunction is:

\begin{equation}
\psi(\mathbf{r}) = \psi_0(\mathbf{r}) + f(\theta) \frac{e^{ikr}}{r}
\end{equation}

where $f(\theta)$ is the scattering amplitude. The radial solution for the partial waves is:

\begin{equation}
R_l(r) \sim \sin\left(kr - \frac{l\pi}{2} + \delta_l\right) \citep{ashcroft1976solid}
\end{equation}

Phase shifts $\delta_l$ are key to calculating the induced electron density and the Friedel sum rule.

\subsubsection{Friedel Sum Rule}
The Friedel sum rule relates the total number of electrons displaced by the impurity to the sum of phase shifts:

\begin{equation}
N_{\rm tot} = \frac{1}{\pi} \sum_{l} (2l+1) \delta_l(k_F) \citep{friedel1952electrons}
\end{equation}

This follows from the density of states modification due to scattering:

\begin{equation}
\Delta \rho(E) = \frac{1}{\pi} \frac{d}{dE} \sum_l (2l+1) \delta_l(E) \citep{newns1973electron}
\end{equation}

Integrating up to $E_F$ yields the total number of electrons affected by the impurity.

\subsubsection{Spatial Dependence of Oscillations}
In 3D systems, oscillations decay as $1/r^3$; in 2D as $1/r^2$; and in 1D as $1/r$ \citep{yafet1987ruderman}. The general expression is:

\begin{equation}
\Delta \rho(r) \approx -\frac{k_F^2}{2\pi^3 r^3} \sum_l (2l+1) \sin(2k_F r - l\pi + 2\delta_l) \citep{ashcroft1976solid}
\end{equation}

\subsection{Experimental Observations}

\subsubsection{Friedel Oscillations in Metals}
Direct experimental observation is challenging due to small amplitudes. Techniques include:

\begin{itemize}
    \item \textbf{X-ray and neutron scattering:} Early indirect evidence from diffraction pattern changes in metals like aluminum and copper \citep{mott1951electronic}.
    \item \textbf{NMR and Mössbauer spectroscopy:} Hyperfine interactions reveal oscillatory variations in the local electronic environment \citep{jaccarino1967hyperfine}.
    \item \textbf{STM:} Modulations around subsurface impurities in sodium and potassium \citep{crommie1993imaging}.
\end{itemize}

\subsubsection{Observations in Alloys}
In alloys, Friedel oscillations affect electronic and structural properties:

\begin{itemize}
    \item \textbf{Cu-Ni and Cu-Fe alloys:} Diffuse X-ray scattering shows oscillatory correlation functions \citep{bragg1954diffuse}.
    \item \textbf{Al-Mg alloys:} Oscillations influence impurity clustering and precipitation \citep{ray1985alloy}.
    \item \textbf{Magnetic alloys (Cu-Mn, Au-Fe):} Oscillatory exchange coupling consistent with RKKY interactions \citep{ruderman1954indirect}.
\end{itemize}

\subsection{Comparative Analysis of Oscillation Characteristics}

\begin{table}[h!]
\centering
\caption{Oscillation characteristics across different metals and alloys}
\begin{tabular}{lcccc}
\toprule
Material / Alloy & $k_F$ (\AA$^{-1}$) & $\lambda = \pi/k_F$ (\AA) & Decay & Notable Observations \\
\midrule
Sodium (Na) & 0.90 & 3.49 & $1/r^3$ & STM density modulations \\
Potassium (K) & 0.75 & 4.19 & $1/r^3$ & NMR evidence \\
Copper (Cu) & 1.36 & 2.31 & $1/r^3$ & Impurity-induced oscillations \\
Aluminum (Al) & 1.75 & 1.79 & $1/r^3$ & Diffuse X-ray scattering \\
Cu-Ni alloys & 1.36 & 2.31 & $1/r^3$ & Short-range order effects \\
Cu-Fe alloys & 1.36 & 2.31 & $1/r^3$ & RKKY interactions \\
Au-Fe alloys & 1.20 & 2.62 & $1/r^3$ & Magnetic coupling \\
Al-Mg alloys & 1.75 & 1.79 & $1/r^3$ & Impurity clustering \\
\bottomrule
\end{tabular}
\end{table}
\begin{figure}[h!]
    \centering
    \input{1literature-review_1friedel_oscillation_decay_plot.tex}
    \caption{Decay of Friedel oscillations for different metals. The wavelength $\lambda = \pi/k_F$ is annotated for each metal.}
    \label{fig:friedel_decay}
\end{figure}
To visualize the differences in Friedel oscillations across various metals, Figure~\ref{fig:friedel_decay} shows the spatial decay of electron density modulation as a function of distance from a single impurity. The oscillations follow the theoretical $1/r^3$ decay in three dimensions, with the wavelength $\lambda = \pi/k_F$ annotated for each metal. This plot highlights how smaller Fermi wavevectors correspond to longer oscillation wavelengths, while larger Fermi wavevectors produce more rapidly oscillating density modulations. The visualization complements Table~\ref{tab:oscillation_characteristics} by providing an immediate, intuitive understanding of both amplitude decay and oscillation frequency in different materials.

	​

 annotated for each metal. This plot highlights how smaller Fermi wavevectors correspond to longer oscillation wavelengths, while larger Fermi wavevectors produce more rapidly oscillating density modulations. The visualization complements the comparative table by providing an immediate, intuitive understanding of both amplitude decay and oscillation frequency in different materials.
\subsection{Advanced Developments and Extensions}

\begin{itemize}
    \item \textbf{Anisotropic Fermi surfaces:} Direction-dependent oscillations \citep{ziman1960electrons}.
    \item \textbf{Surface and low-dimensional systems:} Enhanced oscillations in 2D; $1/r^2$ decay \citep{crommie1993imaging}.
    \item \textbf{Strong impurities:} Require Green's function and DFT approaches \citep{newns1973electron}.
    \item \textbf{Correlation effects:} Modify screening behavior and amplitude \citep{yafet1987ruderman}.
\end{itemize}

\subsection{Summary and Outlook}
Friedel oscillations are a clear manifestation of quantum interference in metals, connecting scattering theory, electron density modulation, and impurity-induced phenomena. Classical theory, via phase shift analysis and the Friedel sum rule, accurately predicts amplitude, wavelength, and decay. Experimental studies in metals and alloys confirm these predictions. Comparative analysis shows universality in physics but sensitivity to material type, impurity strength, and dimensionality. Extensions to complex systems and low-dimensional materials illustrate the continuing relevance of Friedel oscillations in condensed matter physics.

\bibliographystyle{plainnat}
\bibliography{"1literature-review_1-friede-oscilations-bulk.bib"}

\end{document}
